%%%%
% MTecknology's Resume
%%%%
% Author: Michael Lustfield
% License: CC-BY-4
% - https://creativecommons.org/licenses/by/4.0/legalcode.txt
%%%%

\documentclass[letterpaper,10pt]{article}
%%%%%%%%%%%%%%%%%%%%%%%
%% BEGIN_FILE: mteck.sty
%% NOTE: Everything between here and END_FILE can
%% be relocated to "mteck.sty" and then included with:
%\usepackage{mteck}

% Dependencies
% NOTE: Some packages (lastpage, hyperref) require second build!
\usepackage[empty]{fullpage}
\usepackage{titlesec}
\usepackage{enumitem}
\usepackage[hidelinks]{hyperref}
\usepackage{fancyhdr}
\usepackage{fontawesome5}
\usepackage{multicol}
\usepackage{bookmark}
\usepackage{lastpage}
\usepackage{indentfirst}

% Sans-Serif
%\usepackage[sfdefault]{FiraSans}
%\usepackage[sfdefault]{roboto}
%\usepackage[sfdefault]{noto-sans}
%\usepackage[default]{sourcesanspro}

% Serif
\usepackage{helvet}
\usepackage{charter}

% Colors
% Use with \color{Name}
% Can wrap entire heading with {\color{accent} [...] }
\usepackage{xcolor}
\definecolor{accentTitle}{HTML}{005a99}
\definecolor{accentText}{HTML}{005a99}
\definecolor{accentLine}{HTML}{005a99}

% Misc. Options
\pagestyle{fancy}
\fancyhf{}
\fancyfoot{}
\renewcommand{\headrulewidth}{0pt}
\renewcommand{\footrulewidth}{0pt}
\urlstyle{same}

% Adjust Margins
\addtolength{\oddsidemargin}{-0.7in}
\addtolength{\evensidemargin}{-0.5in}
\addtolength{\textwidth}{1.19in}
\addtolength{\topmargin}{-0.7in}
\addtolength{\textheight}{1.4in}

\setlength{\multicolsep}{-3.0pt}
\setlength{\columnsep}{-1pt}
\setlength{\tabcolsep}{0pt}
\setlength{\footskip}{3.7pt}
\raggedbottom
\raggedright

% ATS Readability
%\input{glyphtounicode}
%\pdfgentounicode=1

%-----------------%
% Custom Commands %
%-----------------%

% Centered title along with subtitle between HR
% Usage: \documentTitle{Name}{Details}
\newcommand{\documentTitle}[2]{
  \begin{center}
    {\Huge\color{accentTitle} #1}
    \vspace{10pt}
    {\color{accentLine} \hrule}
    \vspace{2pt}
    %{\footnotesize\color{accentTitle} #2}
    \footnotesize{#2}
    \vspace{2pt}
    {\color{accentLine} \hrule}
  \end{center}
}

% Create a footer with correct padding
% Usage: \documentFooter{\thepage of X}
\newcommand{\documentFooter}[1]{
  \setlength{\footskip}{10.25pt}
  \fancyfoot[C]{\footnotesize #1}
}

% Simple wrapper to set up page numbering
% Usage: \numberedPages
% WARNING: Must run pdflatex twice!
\newcommand{\numberedPages}{
  \documentFooter{\thepage/\pageref{LastPage}}
}

% Section heading with horizontal rule
% Usage: \section{Title}
\titleformat{\section}{
  \vspace{-5pt}
  \color{accentText}
  \raggedright\large\bfseries
}{}{0em}{}[\color{accentLine}\titlerule]

% Section heading with separator and content on same line
% Usage: \tinysection{Title}
\newcommand{\tinysection}[1]{
  \phantomsection
  \addcontentsline{toc}{section}{#1}
  {\large{\bfseries\color{accentText}#1} {\color{accentLine} |}}
}

% Indented line with arguments left/right aligned in document
% Usage: \heading{Left}{Right}
\newcommand{\heading}[2]{
  \hspace{10pt}#1\hfill#2\\
}

% Adds \textbf to \heading
\newcommand{\headingBf}[2]{
  \heading{\textbf{#1}}{\textbf{#2}}
}

% Adds \textit to \heading
\newcommand{\headingIt}[2]{
  \heading{\textit{#1}}{\textit{#2}}
}

% Template for itemized lists
% Usage: \begin{resume_list} [items] \end{resume_list}
\newenvironment{resume_list}{
  \vspace{-7pt}
  \begin{itemize}[itemsep=-2px, parsep=1pt, leftmargin=30pt]
}{
  \end{itemize}
  %\vspace{-2pt}
}

% Formats an item to use as an itemized title
% Usage: \itemTitle{Title}
\newcommand{\itemTitle}[1]{
  \item[] \underline{#1}\vspace{4pt}
}

%% END_FILE: mteck.sty
%%%%%%%%%%%%%%%%%%%%%%


%===================%
% Gujju Gurunadha Reddy's Resume %
%===================%

%\numberedPages % NOTE: lastpage requires a second build
%\documentFooter{\thepage of 2} % Does similar without using lastpage
\begin{document}

  %---------%
  % Heading %
  %---------%

  \documentTitle{Gujju Gurunadha Reddy}{
    % Web Version
    %\raisebox{-0.05\height} \faPhone\ [redacted - web copy] ~
    %\raisebox{-0.15\height} \faEnvelope\ [redacted - web copy] ~
    %%
    \href{tel:+33 759381024}{
      \raisebox{-0.05\height} \faPhone\ +33 759-381-024} ~ | ~
    \href{mailto:gujju.gurunadha.reddy@cern.ch}{
      \raisebox{-0.15\height} \faEnvelope\ gujju.gurunadha.reddy@cern.ch} ~ | ~
    \href{https://linkedin.com/in/gujju-gurunadha-reddy-3073972ab/}{
      \raisebox{-0.15\height} \faLinkedin\ linkedin} ~ | ~
    \href{https://github.com/greddy36}{
      \raisebox{-0.15\height} \faGithub\ github}
  }

  %---------%
  % Summary %
  %---------%

  \tinysection{Summary}
    Ph.D. candidate in Experimental High Energy Physics with expertise in data analysis and pixel detector operation and maintenance as part of the CMS experiment at CERN.
  %--------%
  % Skills %
  %--------%

  \section{Education}
  \begin{itemize}
  \item \headingBf{Kansas State University}{Manhattan, KS}
  \headingIt{PhD in Experimental High Energy Physics}{August 2019 -- Present}
  \heading{GPA - 3.65/4.00}{}
  \item \headingBf{National Institute of Science Education and Research}{Odisha, India}
  \headingIt{Bachelors in Physics}{June 2014 -- May 2019}
  \heading{Grade percentage - 68.6\% }{}
  \end{itemize}

  %------------%
  % Experience %
  %------------%

  \section{Professional Experience}

  \headingBf{Research Assistant, CMS Experiment at LHC}{August 2021 -- Present}
  \headingIt{Kansas State University}{}
  \hangindent=1em
  \hangafter=0
  In the Compact Muon Solenoid (CMS) group at Kansas State University, I have performed extensive physics, statistical and computational analyses of Large Hadron Collider (LHC) data and also assisted in pixel detector operations in Run-3.  
  \begin{itemize}
      \item \textbf{Search for pair produced $H^{\pm\pm}$ in a multi-lepton final state with the CMS detector at $\sqrt{s} =$ 13 TeV}
      \newline
      Designed and implemented a custom mass reconstruction algorithm for 4-lepton (including $\tau)$ plus up to 4-neutrino system in heavy di-resonance searches, utilizing the observed collinearity of neutrino and lepton momenta to approximate invariant masses and improve signal sensitivity.
      
      \item \textbf{Search for the single production of a vector-like
quark decaying into a $W^{\pm}$ boson and a $b$-quark using single lepton final state with the CMS detector at $\sqrt{s} =$ 13 TeV}
      \newline
      Using a recursive-jigsaw mass reconstruction to optimize event kinematics, trained a feed-forward deep neural network (DNN) for signal–background separation. A genetic optimization–based DNN was used to optimize the shape systematics for limit calculations. 
      
      \item \textbf{Pixel Operations}
      \newline
      Designed various tools to assist shifters and experts in diagnosing pixel electronics. Responsible for developing and maintaining a FED (Front-End Driver) monitoring tool that fetches XDAQ flashlists in real-time to generate an error summary. Worked on a fast, firmware based reprogramming scheme for pixel Front-End Chip (FEC). It's estimated to reduce the reconfiguration time from 2 minutes to 4s. Contributed to detector operations by serving as a detector on-call (DOC) and data acquisition expert on a regular basis.

      \item \textbf{FPGA testing for CMS upgrade}
      \newline
      During my time at Kansas State University, I assisted the engineers working on Pixel Upgrade to test the FMC mezzanine cards. It's widely used in the inner tracker upgrade.
  \end{itemize}
  

  \headingBf{Undergraduate Research}{August 2016 -- May 2019}
  \headingIt{National Institute of Science Education and Research}{}
  \begin{itemize}
    \item \textbf{Exotic decays of 125 $GeV$ Higgs to a pair of pseudo-scalars in a $2\gamma2j$ final state with the CMS detector at $\sqrt{s} =$ 13 TeV}
    \newline
    Used a ascending and descending cumulative binning approach to maximize the significance of kinematic cuts and also used a data driven method to model jet backgrounds. 
    \item \textbf{Lock-in amplifier}
    \newline
    Designed and built a lock-in amplifier for extracting weak signals from noisy environments, demonstrating precise phase-sensitive detection. Successfully reconstructed faint audio signals with high signal-to-noise improvement.
  \end{itemize}


  
  \section{Positions of Responsibility}
  \begin{itemize}
    \item  \headingBf{L3 Convener, Pixel DAQ convener}{June 2025 -- Present}
    \headingIt{Pixel Operations, CMS collaboration}{CERN, CH}
    \heading{I oversee the operation, monitoring, and maintenance of the CMS Pixel Data Acquisition system, ensuring reliable}{}
    \heading{performance of both the front-end electronics and the associated control software infrastructure.}{}
    \item  \headingBf{Pixel DAQ expert}{June 2023 -- Present}
    \headingIt{Pixel Operations, CMS collaboration}{CERN, CH}
    \heading{Trained several DOCs and DAQs in pixel operations, reduced operation downtime by developing error monitoring tools}{}
    \heading{and provided technical support for the maintenance of software and electronics}{}
    \end{itemize}
  \section{Talks and Presentations}
   \begin{itemize}
   \item \headingBf{Tracker Week 2025}{CERN, CH}
   \heading{Plenary talk on Pixel Status Report}{November 2025}
   \item \headingBf{Posters@LHCC 2024}{CERN, CH}
   \heading{Performance and Operation of CMS Pixel Tracker in 2023 and 2024}{November 2024}
   \item \headingBf{Special B2G VHF meeting: ATLAS+CMS discussion on VLQ signal MC for Run-3}{Virtual}
   \heading{Presented my findings on the discrepancies between VLQ couplings used by CMS and ATLAS}{February 2024}
   \item \headingBf{JuliaHEP workshop 2023}{Erlangen, DE}
   \heading{Talk given on Using Symbolic Regression in Julia to find analytic functions to model low statistics}{November 2023}
   \item \headingBf{Posters@ESHEP 2022}{Jerusalem, IL}
   \heading{Search for the single production of a vector-like quark $Y/T\rightarrow Wb$ in lepton+jets final state}{December 2022}
   \end{itemize}

   \section{Awards and Honors}
   \begin{itemize}
       \item \headingBf{INSPIRE Scholarship}{June 2014 -- May 2019}
       \headingIt{Department of Physics}{NISER, India}
       \heading{Awarded by the Department of Science and Technology for supporting young talent in fundamental sciences. }{}
   \end{itemize}
   
    \section{Outreach and Teaching}
   \begin{itemize}
       \item \headingBf{Teaching Assistant}{August 2019- August 2021}
       \headingIt{Kansas State University}{}
       \heading{Taught classes and organized labs for PHY 113 (classical mechanics and thermodynamics) and PHY 114 (electromag-}{}
       \heading{netism, optics and modern physics)}{}
       \item \headingBf{Student Assistant}{August 2019 - August 2021}
       \headingIt{Kansas State University}{}
       \heading{Graded and proctored exams for 1st and 2nd year Engineering physics (EP-1 and EP-2)}{}
       \item \headingBf{Tutoring}{January 2020 - May 2022}
       \headingIt{Kansas State University}{}
       \heading{Served as college employed help room tutor for math and physics courses across various departments}{}
       \item \headingBf{HEP outreach}{January 2022}
       \headingIt{Kansas State University}{}
       \heading{Helped in setting up a hands on Higgs discovery exercise for undergraduates using CERN open data}{}
       \item \headingBf{Annual Science Day}{2015, 2016, 2017}
       \headingIt{NISER, India}{}
       \heading{Actively participated in by designing and demonstrating interactive physics experiments for public engagement}{}
   \end{itemize}

   \section{Software and Hardware skills}
   \begin{itemize}
       \item \headingBf{Programming languages}{}
       \heading{Proficient in: C, C++, ROOT, Python, Julia, Shell scripting, MATLAB, Mathematica, TeX, HTML, JavaScript}{}
       \item \headingBf{Code Management}{}
       \heading{Git, VIM, Nano, VSCode}{}
   \end{itemize}
\end{document}
